\documentclass[9pt]{article}

\usepackage[utf8]{inputenc}   % LaTeX, comprends les accents !
\usepackage[T1]{fontenc}      % Police contenant les caractères français
\usepackage[francais]{babel}  % Placez ici une liste de langues, la
                              % dernière étant la langue principale
\usepackage{lipsum}

\usepackage{multicol}
\usepackage{amsmath}
\usepackage{amsfonts}
\usepackage{amssymb}
\usepackage{calrsfs}
\usepackage[mathscr]{euscript}

\setlength{\parindent}{0cm} %No indentation

\usepackage[paperwidth=600px,
            paperheight=450px,
            top=10px,
            bottom=10px,
            left=10px,
            right=10px]{geometry}

% \usepackage[a4paper]{geometry}% Réduire les marges
\pagestyle{headings}        % Pour mettre des entètes avec les titres
                              % des sections en haut de page

% \title{Lorem ipsum}           % Les paramètres du titre : titre, auteur, date
% \author{Curabitur \and Elementum}
% \date{}                       % La date n'est pas requise (la date du
                              % jour de compilation est utilisée en son
			      % absence)

% \sloppy                       % Ne pas faire déborder les lignes dans la marge
\begin{document}
\begin{multicols*}{2}
\part*{Formule}
\section{Décomposition}
$\rightarrow$ cos(a+b) = cos(a)cos(b) - sin(a)sin(b)\\
$\rightarrow$ cos(a-b) = cos(a)cos(b) + sin(a)sin(b)\\
$\rightarrow$ sin(a+b) = sin(a)cos(b) + sin(b)cos(a)\\
$\rightarrow$ sin(a-b) = sin(a)cos(b) - sin(b)cos(a)\\
$\rightarrow$ tan(a+b) = $\dfrac{tan(a) + tan(b)}{1 - tan(a)tan(b)}$\\
$\rightarrow$ tan(a-b) = $\dfrac{tan(a) - tan(b)}{1 + tan(a)tan(b)}$

\section{Angle double}
$\rightarrow$ sin(2a) = 2sin(a)cos(a)\\
$\rightarrow$ cos(2a) = $2cos^2(a)-1$ = $1 - 2sin^2(a)$\\
$\rightarrow$ tan(2a) = $\dfrac{2tan(a)}{1-tan^2(a)}$
\section{Linéarisation}
$\rightarrow$ 2cos(a)cos(b) = cos(a+b) + cos(a-b)\\
$\rightarrow$ 2sin(a)sin(b) = cos(a-b) - cos(a+b)\\
$\rightarrow$ 2sin(a)cos(b) = sin(a+b) + sin(a-b)\\
$\rightarrow$ $cos^2(a) = \dfrac{1+cos(2a)}{2}$\\
$\rightarrow$ $sin^2(a) = \dfrac{1-cos(2a)}{2}$
\section{Somme}
$\rightarrow cos(p) + cos (q) = 2(cos(\dfrac{p+q}{2}).cos(\dfrac{p-q}{2}))$ \\
$\rightarrow$cos(p) - cos (q) = -$2(sin(\dfrac{p+q}{2}).sin(\dfrac{p-q}{2}))$\\
$\rightarrow$sin(p) + sin (q) = $2(sin(\dfrac{p+q}{2}).cos(\dfrac{p-q}{2}))$\\
$\rightarrow$sin(p) - sin (q) = $2(cos(\dfrac{p+q}{2}).sin(\dfrac{p-q}{2}))$
\section{Arc moitié}
On pose $t=tan(\frac{\theta}{2})$ on a\\
$cos(a) = \frac{1-t^2}{1+t^2} \qquad sin(a) = \frac{2t}{1+t^2} \qquad tan(a) = \frac{2t}{1-t^2}$\\
$arcsin(x) + arccos(x) = \frac{\pi}{2}$\\
$arctan(x) + arctan(\frac{1}{x})$ = $\pm\dfrac{\pi}{2}$ (dépend du signe de x)\\
$cos^2(x)+sin^2(x) = 1 \qquad ch^2(x) - sh^2(x) = 1$
\part*{Fonction inverse}
\section{Fonction Hyperbolique}
\begin{center}
  \begin{tabular}{|l|l|l|l|} \hline
    Fonction & $D_f$ & $D_{f'}$ & f'(x) \\ \hline
    Argch & $]1:+\infty$ & ]1;$+\infty$[ & $\dfrac{1}{\sqrt{x^2-1}}$ \\   \hline
    Argsh & $\Re$ & $\Re$ & $\dfrac{1}{\sqrt{x^2+1}}$ \\  \hline
    Argth & $]-1;1[$ & $]-1;1[$ & $\dfrac{1}{1-x^2}$ \\  \hline
  \end{tabular}
\end{center}
\section{Fonction Trigonométrique}
\begin{center}
% use packages: array
\begin{tabular}{|l|l|l|l|}
\hline
Fonction & $D_f$ & $D_{f'}$ & f'(x) \\ \hline
Arccos & [-1:1] & ]-1;1[ & $\dfrac{-1}{\sqrt{1-x^2}}$ \\  \hline
Arcsin & [-1:1] & ]-1;1[ & $\dfrac{1}{\sqrt{1-x^2}}$ \\ \hline
Arctan & $\Re$ & $\Re$ & $\dfrac{1}{1+x^2}$ \\ \hline
\end{tabular}
\end{center}
% \raggedcolumns


\end{multicols*}


\end{document}