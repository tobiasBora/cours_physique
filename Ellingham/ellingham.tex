% Compiler avec
% latex fig.tex && convert -alpha Remove -resize 800x600 -depth 8 fig.dvi fig3.png

% \documentclass[convert={density=300,size=1080x800,outext=.png}]{standalone}
\documentclass[9pt]{article}

\usepackage[utf8]{inputenc}   % LaTeX, comprends les accents !
\usepackage[T1]{fontenc}      % Police contenant les caractères français
\usepackage[francais]{babel}  % Placez ici une liste de langues, la
                              % dernière étant la langue principale
\usepackage{lipsum}

\usepackage{multicol}
\usepackage{amsmath}
\usepackage{amsfonts}
\usepackage{amssymb}
\usepackage{calrsfs}
\usepackage[mathscr]{euscript}


\usepackage[paperwidth=600px,
            paperheight=450px,
            top=10px,
            bottom=10px,
            left=10px,
            right=10px]{geometry}

% \usepackage[a4paper]{geometry}% Réduire les marges
\pagestyle{headings}        % Pour mettre des entêtes avec les titres
                              % des sections en haut de page

% \title{Lorem ipsum}           % Les paramètres du titre : titre, auteur, date
% \author{Curabitur \and Elementum}
% \date{}                       % La date n'est pas requise (la date du
                              % jour de compilation est utilisée en son
			      % absence)

% \sloppy                       % Ne pas faire déborder les lignes dans la marge

\begin{document}

\begin{multicols*}{2}
\setlength{\columnseprule}{0.1pt}
\part{Diagrammes d'Ellingham}
\section{Construction du diagramme}
On étudie la réaction entre deux couples oxydes/métaux. Le diagramme consiste à tracer $\Delta_rG^0$ en fonction de $T$. Pour calculer $\Delta_rG^0$ on se place dans l'approximation d'Ellingham avec $\Delta_rH^0$ et $\Delta_rS^0$ de constant tant qu'il n'y a pas de changement de température puis on utilise la formule:
$$\Delta_rG^0 = \Delta_rH^0 - T\Delta_rS^0$$

Si on produit plus de gaz qu'en n'en consomme la pente est positive ($\Delta_rS > 0$) et vis versa.

Lorsque le métal est porté à plus haute température et change d'état, la pente augment.

Le métal est à placer sous la pente tandis que l'oxyde est à placer au dessus

Pour savoir dans quelle zone on se trouve il suffit de comparer la pression avec la pression d'équilibre. Si $P > P_{eq}$ c'est l'oxyde qui est stable, si $P < P_{eq}$ c'est le métal (on introduit la variable $y = -RT_1ln(Q)$ qu'il faut comparer avec $\Delta_rG^0$).

On a corrosion lorsque $P > P_c$ : la pression de corrosion augmente avec T. Elle est souvent très faible ($10^{-88}$ bar pour le fer) sauf pour les métaux nobles (Or : $10^{15}$ bar). => Les métaux les plus facilement oxydables sont dans le bas du diagramme et les métaux nobles dans le haut du diagramme.
\\


Un oxyde est réduit par tout métal dont la droite d'Ellingham est en dessous de la sienne.

\footnotesize{Fiches créés par Alexis Ducarouge et Léo Colisson.}

% \raggedcolumns
\end{multicols*}

\end{document}

