% Compiler avec
% latex fig.tex && convert -alpha Remove -resize 800x600 -depth 8 fig.dvi fig3.png

% \documentclass[convert={density=300,size=1080x800,outext=.png}]{standalone}
\documentclass[9pt,twocolumn]{article}

\usepackage[utf8]{inputenc}   % LaTeX, comprends les accents !
\usepackage[T1]{fontenc}      % Police contenant les caractères français
\usepackage[francais]{babel}  % Placez ici une liste de langues, la
                              % dernière étant la langue principale
\usepackage{lipsum}

\usepackage{multicol}
\usepackage{amsmath}
\usepackage{amsfonts}
\usepackage{amssymb}
\usepackage{calrsfs}
\usepackage{mathrsfs}
\usepackage[mathscr]{euscript}


\usepackage[paperwidth=600px,
            paperheight=450px,
            top=10px,
            bottom=10px,
            left=10px,
            right=10px]{geometry}

\newcommand\blfootnote[1]{%
  \begingroup
  \renewcommand\thefootnote{}\footnote{#1}%
  \addtocounter{footnote}{-1}%
  \endgroup
}
% \usepackage[a4paper]{geometry}% Réduire les marges
\pagestyle{headings}        % Pour mettre des entêtes avec les titres
                              % des sections en haut de page

% \title{Lorem ipsum}           % Les paramètres du titre : titre, auteur, date
% \author{Curabitur \and Elementum}
% \date{}                       % La date n'est pas requise (la date du
                              % jour de compilation est utilisée en son
			      % absence)

% \sloppy                       % Ne pas faire déborder les lignes dans la marge

\begin{document}

\setlength{\columnseprule}{0.1pt}
\begin{center}
\part*{Dipôles non linéaires}
\end{center}
\section{Diode}

Sa caractéristique (graphe $i=f(u)$) est non linéaire.

Lorsque l'intensité est nulle dans la diode, on dit qu'elle est \textbf{bloquée ($\sim$ coupe circuit}), et lorsqu'elle est non nulle, on dit qu'elle est \textbf{passante ($\sim$ fil)}.
\medbreak
Pour connaître l'état de la diode, tracer la courbe $i=f(u_{diode})$ pour le circuit (droite de charge) sur la caractéristique de la diode, leur intersection étant le \textbf{point de fonctionnement}.
\smallbreak
Cela revient à comparer la tension appliquée à la diode à sa tension de seuil.

\medbreak 

$\Rightarrow$ Pour les problèmes en comprenant : 

$\>\>\>\>$- trouver les instants de (dé)blocage de la diode.

$\>\>\>\>$- modéliser les circuits équivalents en fonction.



\section{Oscillateurs quasi-sinusoïdaux}

Principe : les équations des oscillateurs contiennent toujours un terme dissipatif, l'idée est d'ajouter un montage apportant de l'énergie (A.O.) pour rendre cette résistance négative.

% \raggedcolumns


\section{A.O. en régime non linéaire}

\textbf{A.O. parfait :} $\boxed{i_+=i_-=0}$

\textbf{A.O. en régime linéaire :} $\boxed{V^+=V^-}$

\medbreak

On a alors : $\>\>\>\>\>\>\boxed{\underline{v_s}=\dfrac{A_0}{1+j\frac{\omega}{\omega_c}} (V^+-V^-)}\>\>\>$

 avec $A_0>10^5$ et $\omega_c\simeq 100 \>rad.s^{-1}$.
 
 Il faut étudier l'équation différentielle du montage pour conclure quant à sa nature mais :
 
 S'il n'y a pas de bouclage entre l'entrée inverseuse (-) et la sortie, seul le fonctionnement non linéaire est possible.
 
 \medbreak
 Si comportement non linéaire :
 
  - $V_s=+V_{sat}\Leftrightarrow V_+>V_-$
  
  - $V_s=-V_{sat}\Leftrightarrow V_+<V_-$
  \medbreak
  Pour les montages à \textbf{hystérésis} : déterminer quand $V_s=\pm V_{sat}$ en fonction de $V_e$.

% \raggedcolumns


\blfootnote{Fiches créés par Alexis Ducarouge et Léo Colisson.}


\end{document}

