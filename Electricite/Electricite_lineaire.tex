
\documentclass[9pt]{article}

\usepackage[utf8]{inputenc}   % LaTeX, comprends les accents !
\usepackage[T1]{fontenc}      % Police contenant les caractères français
\usepackage[francais]{babel}  % Placez ici une liste de langues, la
                              % dernière étant la langue principale
\usepackage{lipsum}

\usepackage[european resistors]{circuitikz}
\usepackage{sistyle}

\usepackage{multicol}
\usepackage{amsmath}
\usepackage{amsfonts}
\usepackage{amssymb}
\usepackage{calrsfs}
\usepackage[mathscr]{euscript}


\usepackage[paperwidth=600px,
            paperheight=450px,
            top=10px,
            bottom=10px,
            left=10px,
            right=10px]{geometry}

\pagestyle{headings}        % Pour mettre des entêtes avec les titres
                              % des sections en haut de page

\begin{document}

% Pour avoir deux colonnes
\begin{multicols*}{2}
\setlength{\columnseprule}{0.1pt}



\part*{Électricité : filtres linéaires}
\section{Ordres de grandeurs}
Impédance de sortie d'un G.B.F : $R_s = 50 \ohm$

Impédance d'entrée d'un oscilloscope :

Couplage DC :
\shorthandoff{:!}
\begin{circuitikz}[scale=0.7]\draw
(0,0) -- (4,0)
      to[C, l_=\SI{15}{pF}] (4,2)
      -- (0,2)
(2,0) to[R, l^=\SI{1}{M\ohm}] (2,2)
;
\end{circuitikz}
\shorthandon{:!}

Couplace AC :
\shorthandoff{:!}
\begin{circuitikz}[scale=0.7]\draw
(0,0) -- (4,0)
      to[C, l_=$C_e$] (4,2)
      -- (2,2)
      to[C, l^=\SI{0.1}{\micro F}] (0,2)
(2,0) to[R, l_=$R_e$] (2,2)
;
\end{circuitikz}
\shorthandon{:!}

Capacité linéïque câble coaxial : $100pF.m^{-1}$
\section{Deuxième ordre}
\subsection{Passe-bas}

Forme canonique : $\>$ $\>$  $\>$ $\underline{H}(j\omega) = \dfrac{H_{0}}{ 1-(\frac{\omega}{\omega_{0}})^{2}+j\frac{\omega}{Q\omega_{0}} }$\\


Pulsation de résonance par étude de la dérivée du module :


$\>$ $\omega_{r} = \omega_{0}\sqrt{1-\frac{1}{2Q^{2}}} \>$    ssi    $\> Q > \frac{1}{\sqrt{2}}$ $\>$ $\>$ $\>$ $\>$ $\>$ $\>$ $\>$ 
on a : $\Phi\in[0,-\pi]$

\subsection{Passe-haut}

Forme canonique :$\>$ $\>$ $\>$ $\underline{H}(j\omega) = \dfrac{H_{0}(-(\frac{\omega}{\omega_{0}})^{2})}{ 1-(\frac{\omega}{\omega_{0}})^{2}+j\frac{\omega}{Q\omega_{0}} }$\\


Pulsation de résonance par étude de la dérivée du module :


$\>$ $\omega_{r} = \frac{\omega_{0}}{\sqrt{1-\frac{1}{2Q^{2}}}} \>$    ssi    $\> Q > \frac{1}{\sqrt{2}}$ $\>$ $\>$ $\>$ $\>$ $\>$ $\>$ $\>$ $\>$ $\>$ $\>$ $\>$ $\>$ 
on a : $\Phi\in[\pi,0]$


\subsection{Passe-bande}

Forme canonique :

$\>$ $\underline{H}(j\omega) = \dfrac{H_{0}((\frac{\omega}{Q\omega_{0}}))}{ 1-(\frac{\omega}{\omega_{0}})^{2}+j\frac{\omega}{Q\omega_{0}} } 
= \dfrac{H_{0}}{1 + jQ(\frac{\omega}{\omega_{0}}-\frac{\omega_{0}}{\omega})} $ \\


Pulsation de résonance par résolution d'équation de degré 4 obtenue par dérivation du module de la seconde expression :


$\>$ $\omega_{r} = \dfrac{\omega_{0}}{2Q}( \pm 1 + \sqrt{1+4Q^{2}} )$  $\>$ de plus : $\>$ $\Delta\omega=\dfrac{\omega_{0}}{Q}$ et $\Phi\in[\frac{\pi}{2},-\frac{\pi}{2}]$

\subsection{Coupe-bande}

Forme canonique :

$\>$ $\underline{H}(j\omega) = \dfrac{H_{0}(1-(\frac{\omega}{\omega_{0}})^{2})}{ 1-(\frac{\omega}{\omega_{0}})^{2}+j\frac{\omega}{Q\omega_{0}} } 
= \dfrac{H_{0}}{1 + j\frac{x}{Q(1-x^{2})}} $ avec 
$ x = \frac{\omega}{\omega_{0}}$ \\


Étude de la phase, discontinue en 0, par la seconde expression :


$\>$ $tan(\Phi)=\dfrac{x}{Q(1-x^{2})}$ avec $\Phi\in[-\frac{\pi}{2},\frac{\pi}{2}]$


\section{Décomposition de Fourier}
\subsection{Décomposition}

$s(t) = A_{0} + \sum\limits_{k=1}^{+\infty} {A_{k} cos(k \omega t)} + \sum\limits_{k=1}^{+\infty} {B_{k} sin(k \omega t)}$\\

avec : $A_{k} = \frac{2}{T} \int_{t_{0}}^{t_{0}+T} s(t) cos(k \omega t) \, \mathrm{d}t$   $\>$ $\>$ $\>$  $\>$ s impaire $\Rightarrow A_{k} = 0 $

 $\>$ $\>$ $\>$ $\>$ $\>$ $B_{k} = \frac{2}{T} \int_{t_{0}}^{t_{0}+T} s(t) sin(k \omega t) \, \mathrm{d}t$  $\>$  $\>$ $\>$ $\>$ s paire $\Rightarrow B_{k} = 0 $
 
 
\subsection{Aspect énergétique}

$S_{RMS}^{2} = \frac{1}{T} \int_{t_{0}}^{t_{0}+T} s^{2}(t)  \, \mathrm{d}t$

Formule de Parceval : $S_{RMS}^{2} = S^{2}_{0} + \frac{1}{2}\sum\limits_{k=1}^{+\infty} {C^{2}_{k}}$


\footnotesize{Fiches créés par Alexis Ducarouge et Léo Colisson.}


\end{multicols*}

\end{document}