% Compiler avec
% latex fig.tex && convert -alpha Remove -resize 800x600 -depth 8 fig.dvi fig3.png

% \documentclass[convert={density=300,size=1080x800,outext=.png}]{standalone}
\documentclass[9pt,twocolumn]{article}

\usepackage[utf8]{inputenc}   % LaTeX, comprends les accents !
\usepackage[T1]{fontenc}      % Police contenant les caractères français
\usepackage[francais]{babel}  % Placez ici une liste de langues, la
                              % dernière étant la langue principale
\usepackage{lipsum}

\usepackage{multicol}
\usepackage{amsmath}
\usepackage{amsfonts}
\usepackage{amssymb}
\usepackage{calrsfs}
\usepackage{mathrsfs}
\usepackage[mathscr]{euscript}


\usepackage[paperwidth=600px,
            paperheight=450px,
            top=10px,
            bottom=10px,
            left=10px,
            right=10px]{geometry}

% \usepackage[a4paper]{geometry}% Réduire les marges
\pagestyle{headings}        % Pour mettre des entêtes avec les titres
                              % des sections en haut de page

% \title{Lorem ipsum}           % Les paramètres du titre : titre, auteur, date
% \author{Curabitur \and Elementum}
% \date{}                       % La date n'est pas requise (la date du
                              % jour de compilation est utilisée en son
			      % absence)

% \sloppy                       % Ne pas faire déborder les lignes dans la marge

\begin{document}

\setlength{\columnseprule}{0.1pt}
\begin{center}
\part*{Induction}
\end{center}
\section{Méthode : courant induit}


$\bullet$ 1. Orientation du circuit

$\bullet$ 2. Calcul de $e_{m}$ ou $\overrightarrow{E_{m}}$

\hspace{4mm}
   $\overrightarrow{E_{m}}=-\dfrac{\partial\overrightarrow{A}}{\partial t}+\overrightarrow{v_{e}}\wedge\overrightarrow{B}\>\>\>\>\>\>\>\>\>\>\>\>\>\>\>\>\>\>\>\>\>\>\>\>\>\>\>\> \Rightarrow e_{AB}=\displaystyle{\overrightarrow{E_{m}}(M)\cdot dl_{M}}$
   
 \hspace{4mm}  \textbf{Loi de Faraday :} $\boxed{e_{m}=-\dfrac{d \phi_{\overrightarrow{B}}}{dt}} \>\>\>$ avec $\>\>\> \displaystyle{\phi_{\overrightarrow{B}}=\iint \overrightarrow{B}\cdot\overrightarrow{dS}}$


$\bullet$ 3. Circuit équivalent

\hspace{4mm} Prendre en compte : $\>$- résistance 

\hspace{37mm} - f.e.m inductive dans le sens de $\overrightarrow{i}$ 

\hspace{37mm} - auto-inductance

\hspace{4mm} $\hookrightarrow$ Loi des mailles


$\bullet$ 4. Calcul de i


\section{Inductance réciproque}

On note $\phi_{1\rightarrow2}$ le flux du champ $\overrightarrow{B_{1}}$ créé par $i_{1}$ et de même réciproquement.

On appel $M$ le coefficient d'inductance mutuelle (en Henry) :

$\phi_{1\rightarrow2}=Mi_{1} \>\>et\>\> \phi_{2\rightarrow1}=Mi_{2}$


\section{Inductance propre}

Un circuit parcouru par un courant génère un champ magnétique propre $\overrightarrow{B_{p}}$ qui un génère un flux proportionnel à $i$, on défini ainsi l'auto-inductance L : $\>\>\>\>\>\>\>\>\boxed{\phi_{p}=Li}$

d'où $e_p=-\dfrac{d\phi_p}{dt}=-L\dfrac{di}{dt}$  $\>\>\>\>\>\>\>\>\Rightarrow$ ne pas le compter 2 fois !


\section{Énergie}

$E_m=\dfrac{1}{2}Li^2=\dfrac{1}{2}\phi_p i$

Pour un circuit couplé : $\mathscr{P}_{magn} =\dfrac{d} {dt} (\dfrac{1}{2} L_1 i_1^2+\dfrac{1} {2} L_2 i_2^2+Mi_1 i_2)$

\section{Exercice d'induction}

Distinguer : 

- induction de \textbf{Neuman} : champ magnétique en mouvement

- induction de \textbf{Lorentz} : circuit est en mouvement dans un champ magnétique fixe 


$\newline$

Équations utiles : - f.e.m. d'induction par Lorenz 

\hspace{28mm}- électrique (loi des mailles)

\hspace{28mm}- force de Laplace

\hspace{28mm}- mécanique (TRD ou TMC)

\section{Force de Laplace}

$\overrightarrow{F_{lap}}=I\overrightarrow{dl}\wedge\overrightarrow{B}$

\vfill
\footnotesize{Fiches créés par Alexis Ducarouge et Léo Colisson.}

% \raggedcolumns

\end{document}

