% Compiler avec
% latex fig.tex && convert -alpha Remove -resize 800x600 -depth 8 fig.dvi fig3.png

% \documentclass[convert={density=300,size=1080x800,outext=.png}]{standalone}
\documentclass[9pt]{article}

\usepackage[utf8]{inputenc}   % LaTeX, comprends les accents !
\usepackage[T1]{fontenc}      % Police contenant les caractères français
\usepackage[francais]{babel}  % Placez ici une liste de langues, la
                              % dernière étant la langue principale
\usepackage{lipsum}

\usepackage{multicol}
\usepackage{amsmath}
\usepackage{amsfonts}
\usepackage{amssymb}
\usepackage{calrsfs}
\usepackage[mathscr]{euscript}


\usepackage[paperwidth=600px,
            paperheight=450px,
            top=10px,
            bottom=10px,
            left=10px,
            right=10px]{geometry}

% \usepackage[a4paper]{geometry}% Réduire les marges
\pagestyle{headings}        % Pour mettre des entêtes avec les titres
                              % des sections en haut de page

% \title{Lorem ipsum}           % Les paramètres du titre : titre, auteur, date
% \author{Curabitur \and Elementum}
% \date{}                       % La date n'est pas requise (la date du
                              % jour de compilation est utilisée en son
			      % absence)

% \sloppy                       % Ne pas faire déborder les lignes dans la marge

\begin{document}

\begin{multicols*}{2}
\setlength{\columnseprule}{0.1pt}
\part*{Dipôle magnétique}
\section{Moment magnétique. Dipôle magnétique passif}
 Couple exercé sur une spire de surface $S$ par un champ magnétique :
 
 $\boxed{ \Gamma= \overrightarrow{\mathcal{M}}\wedge\overrightarrow{B_{ext}}}$ avec $\overrightarrow{\mathcal{M}}$ le moment magnétique  : $\boxed{\overrightarrow{\mathcal{M}} = IS\overrightarrow{n}}$
 
 Énergie potentielle magnétique : $E_{p}=-\overrightarrow{\mathcal{M}}\cdot\overrightarrow{B_{ext}}$
 
 \section{Situation active : champ à grande distance}
$\overrightarrow{A}$ a les même symétries que $\overrightarrow{j}$ donc est
 $\boxed{\overrightarrow{A(M)}=\dfrac{\mu_{0}\overrightarrow{\mathcal{M}}\wedge\overrightarrow{e_{r}}}{4 \pi r^{2}}}$
 
$\overrightarrow{B}$ avec $\overrightarrow{B}=\overrightarrow{rot}\overrightarrow{A}=\dfrac{\mu_{0}}{4\pi r^{2}}(3(\overrightarrow{\mathcal{M}}\wedge\overrightarrow{e_{r}})\overrightarrow{e_{r}}-\overrightarrow{\mathcal{M}})$

Sur une spire : $\overrightarrow{B_r}=\dfrac{\mu_{0}\cdot(\pi r^{2}I)}{2\pi}\dfrac{sin^{3}\alpha}{r^{3}}$

Expression du champ : 
$\overrightarrow{B}=$
\begin{tabular}{|c}
$\frac{\mu _0}{4\pi }\dfrac{2 \mathcal{M} cos\theta}{r^3}$ \\ 

$\frac{\mu _0}{4\pi }\dfrac{ \mathcal{M} sin\theta}{r^3}$ \\ 

$0$ \\ 
\end{tabular} 

\section{Force de Laplace}

$\overrightarrow{F_{lap}}=I\overrightarrow{dl}\wedge\overrightarrow{B}$


\part*{Dipôle électrostatique}


\section{Expression du champ}

$\overrightarrow{E}=-\overrightarrow{grad}V= $
\begin{tabular}{|c}
$\frac{1}{4\pi \varepsilon _0}\dfrac{2p cos\theta}{r^3}$ \\ 

$\frac{1}{4\pi \varepsilon _0}\dfrac{p sin\theta}{r^3}$ \\ 

$0$ \\ 
\end{tabular} 
\vfill\null
\footnotesize{Fiches créés par Alexis Ducarouge et Léo Colisson.}
% \raggedcolumns
\end{multicols*}

\end{document}

