% Compiler avec
% latex fig.tex && convert -alpha Remove -resize 800x600 -depth 8 fig.dvi fig3.png

% \documentclass[convert={density=300,size=1080x800,outext=.png}]{standalone}
\documentclass[9pt,twocolumn]{article}

\usepackage[utf8]{inputenc}   % LaTeX, comprends les accents !
\usepackage[T1]{fontenc}      % Police contenant les caractères français
\usepackage[francais]{babel}  % Placez ici une liste de langues, la
                              % dernière étant la langue principale
\usepackage{lipsum}

\usepackage{multicol}
\usepackage{amsmath}
\usepackage{amsfonts}
\usepackage{amssymb}
\usepackage{calrsfs}
\usepackage{mathrsfs}
\usepackage[mathscr]{euscript}


\usepackage[paperwidth=600px,
            paperheight=450px,
            top=10px,
            bottom=10px,
            left=10px,
            right=10px]{geometry}

% \usepackage[a4paper]{geometry}% Réduire les marges
\pagestyle{headings}        % Pour mettre des entêtes avec les titres
                              % des sections en haut de page

% \title{Lorem ipsum}           % Les paramètres du titre : titre, auteur, date
% \author{Curabitur \and Elementum}
% \date{}                       % La date n'est pas requise (la date du
                              % jour de compilation est utilisée en son
			      % absence)

% \sloppy                       % Ne pas faire déborder les lignes dans la marge

\begin{document}

\setlength{\columnseprule}{0.1pt}
\begin{center}
\part*{Diffusion thermique}
\end{center}
\section{Modes de transfert thermique}

$\bullet$ \textbf{Conduction :} pas de mouvement macroscopique. C'est un phénomène de conductibilité thermique en particulier pour les milieux solides. Également appelé \textbf{diffusion} thermique.

\medbreak

$\bullet$ \textbf{Convection :} mouvement macroscopique des milieux fluides. Naturelle en cas de gradient de température ou forcée par intervention de l'Homme.

\medbreak

$\bullet$ \textbf{Rayonnement électromagnétique :} ne nécessite pas de support matériel.


\section{Flux d'énergie}

On définit $\overrightarrow{j_{th}}$ vecteur densité de courant thermique.

La puissance thermique traversant $\overrightarrow{dS}$ est $d\phi=\overrightarrow{j_{th}}\cdot\overrightarrow{dS}$

d'où $\boxed{\delta Q=d\phi \> dt=\overrightarrow{j_{th}}\cdot\overrightarrow{dS}\>dt}$

\bigbreak

On considère alors la loi \textbf{phénoménologique} de \textbf{Fourier} : 

\begin{center}$\boxed{\overrightarrow{j_{th}}=-\lambda\>\overrightarrow{grad}T}$
\end{center}

avec $\lambda$ la conductivité thermique, par analogie avec le potentiel électrique V  ($\overrightarrow{j}= \gamma \overrightarrow{E} = - \gamma \overrightarrow{grad} V$), T est le potentiel thermique.

$\newline \newline$

\footnotesize{Fiches créés par Alexis Ducarouge et Léo Colisson.}

\section{Équation de diffusion thermique / de la chaleur}
\begin{center}

Version linéique : $\boxed{ \rho c\dfrac{\partial T}{\partial t}=\lambda\dfrac{\partial^2 T}{\partial x^2}+\mathscr{P}_{v}}$ 
\end{center}

avec $\dfrac{\lambda}{\rho c}$ la diffusivité thermique

Version cylindrique : $ \rho c\dfrac{\partial T}{\partial t}=-\dfrac{\lambda}{r}\dfrac{\partial r j_{th}}{\partial r}$

Version sphérique : $ \rho c\dfrac{\partial T}{\partial t}=-\dfrac{\lambda}{r^2}\dfrac{\partial r^2 j_{th} }{\partial r}$

Dans le cas général : $\boxed{\rho c\dfrac{\partial T}{\partial t}=\lambda \>\Delta T}$
\medbreak
\underline{Méthode d'établissement :} avec le premier principe 

$dU=\rho d\tau c \dfrac{\partial T}{dt}dt$ (loi de Joule)

$\delta Q_e-\delta Q_s=\lbrace[j_{th}S](l)-[j_{th}S](l+dl)\rbrace dt=-\dfrac{\partial j_{th}(l)S(l)}{\partial l}dldt$

puis utiliser la loi de Fourier.

\section{Convection}

\textbf{Loi de Newton :} $\boxed{j_{cc}=h(T(x=0)-T_{\infty})}$

\smallbreak
Condition limite conducto-convective (pour $\frac{\partial T}{\partial x}$) :

$\>\>\>\>\>\>\>\>\>\>\>\>\>\>j_{cc}=h(T(x=0)-T_{\infty})=j_{th}=-\lambda\>\overrightarrow{grad}T$

\section{Résistance thermique}

$\boxed{R_{th}=\frac{1}{\lambda}\frac{l}{S}}\>\>$ Montage en série : traversé par le même flux$_{th}$.
\smallbreak
Mais aussi : $\boxed{R_{th}=\dfrac{T_2-T_1}{\phi_{1\rightarrow 2}}}\>\>\>$ avec : $\displaystyle{\phi_{1\rightarrow 2}= \iint _{S_1} \overrightarrow{j_{th_1}}\cdot\overrightarrow{dS_1}}$


% \raggedcolumns

\end{document}

