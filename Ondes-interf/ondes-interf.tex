% Compiler avec
% latex fig.tex && convert -alpha Remove -resize 800x600 -depth 8 fig.dvi fig3.png

% \documentclass[convert={density=300,size=1080x800,outext=.png}]{standalone}
\documentclass[9pt]{article}

\usepackage[utf8]{inputenc}   % LaTeX, comprends les accents !
\usepackage[T1]{fontenc}      % Police contenant les caractères français
\usepackage[francais]{babel}  % Placez ici une liste de langues, la
                              % dernière étant la langue principale
\usepackage{lipsum}

\usepackage{multicol}
\usepackage{amsmath}
\usepackage{amsfonts}
\usepackage{amssymb}
\usepackage{calrsfs}
\usepackage[mathscr]{euscript}


\usepackage[paperwidth=600px,
            paperheight=450px,
            top=10px,
            bottom=10px,
            left=10px,
            right=10px]{geometry}

% \usepackage[a4paper]{geometry}% Réduire les marges
\pagestyle{headings}        % Pour mettre des entêtes avec les titres
                              % des sections en haut de page

% \title{Lorem ipsum}           % Les paramètres du titre : titre, auteur, date
% \author{Curabitur \and Elementum}
% \date{}                       % La date n'est pas requise (la date du
                              % jour de compilation est utilisée en son
			      % absence)

% \sloppy                       % Ne pas faire déborder les lignes dans la marge

\begin{document}

\begin{multicols*}{2}
\setlength{\columnseprule}{0.1pt}
\part{Les ondes lumineuses}                % Commencer une partie...

\section{Généralités}               % Commencer une section, etc.

Eclairement : $\mathcal{E} = K<||\vec{E}||>$
Approximation scalaire : $\underline{s}(M,T) = s_me^{i(\omega t - \phi(M))}$
avec $\underline{s}=s_me^{-i \phi(O)}$

On a
 $$\phi_{A \rightarrow B} = \frac{2 \pi}{\lambda_0} (AB) + \phi_{sup}$$

Attention un déphasage de $\pi$ apparait lorsqu'il y a réflexion sur un milieu d'indice supérieur ($n_1 < n_2$).

\section{Principe de Fermat}
Entre deux points A et B atteints par la lumière, le chemin optique du trajet suivi par la lumière est stationnaire.

\section{Théorème de Malus}
Les rayons lumineux sont normaux aux surfaces d'ondes.

\section{Résultats à connaître}

\subsection{Lame}
Pour une lame d'épaisseur $e$ on a $\boxed{\delta = 2 n e cos(r)}$

Pour l'obtenir il suffit d'utiliser le théorème de Malus, puis de jouer un peu avec la trigo en n'ayant pas peur d'avoir des divisions par $cos(r)$ qui devraient se simplifier avec la relation de Descartes $n_1sin(i_1)=n_1sin(i_2)$

\part{Interférences à deux ondes}                % Commencer une partie...
\section{Deux sources de même fréquence}
On a avec M le point considéré pour chaque source :
$$s_1 = s_{1m}cos(1\pi \nu t - \phi_{10} - \frac{2\pi (S_1M)}{\lambda_0})$$
La formule de superposition est alors
$$\mathcal{E} = \mathcal{E}_1 + \mathcal{E}_2 + 2 \sqrt{\mathcal{E}_1\mathcal{E}_2}cos(\frac{2 \pi \delta}{\lambda_0} + \Delta\phi)$$
Et dans le cas le plus courant ($\mathcal{E}_1 = \mathcal{E}_2$ et $\Delta\phi = 0$) on a
$$\boxed{\mathcal{E} = 2\mathcal{E}_0 \left(1 + cos(\frac{2 \pi \delta}{\lambda_0})\right)}$$

L'ordre d'interférence est défini par $p = \frac{\delta}{\lambda_0} $ qui correspond au numéro de la frange.

NB : pour calculer $\mathcal{E}$ dans d'autres cas il est conseillé d'utiliser la formule $<s>^2 = \frac{1}{2} (s\cdot s^*)$

\subsection{Formes à connaître}
Dans le cas où les sources sont parallèles à l'écran (trous de Young quand S est centrée par exemple), on a
$\boxed{\delta = n \frac{ax}{D}}$ et l'interfrange est $\boxed{i = \frac{\lambda_0D}{na}}$

(Preuve pas jolie avec un DL)

\footnotesize{Fiches créés par Alexis Ducarouge et Léo Colisson.}

% \raggedcolumns
\end{multicols*}

\end{document}

