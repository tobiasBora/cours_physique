% Compiler avec
% latex fig.tex && convert -alpha Remove -resize 800x600 -depth 8 fig.dvi fig3.png

% \documentclass[convert={density=300,size=1080x800,outext=.png}]{standalone}
\documentclass[9pt]{article}

\usepackage[utf8]{inputenc}   % LaTeX, comprends les accents !
\usepackage[T1]{fontenc}      % Police contenant les caractères français
\usepackage[francais]{babel}  % Placez ici une liste de langues, la
                              % dernière étant la langue principale
\usepackage{lipsum}

\usepackage{multicol}
\usepackage{amsmath}
\usepackage{amsfonts}
\usepackage{amssymb}
\usepackage{calrsfs}
\usepackage[mathscr]{euscript}


\usepackage[paperwidth=600px,
            paperheight=450px,
            top=10px,
            bottom=10px,
            left=10px,
            right=10px]{geometry}

% \usepackage[a4paper]{geometry}% Réduire les marges
\pagestyle{headings}        % Pour mettre des entêtes avec les titres
                              % des sections en haut de page

% \title{Lorem ipsum}           % Les paramètres du titre : titre, auteur, date
% \author{Curabitur \and Elementum}
% \date{}                       % La date n'est pas requise (la date du
                              % jour de compilation est utilisée en son
			      % absence)

% \sloppy                       % Ne pas faire déborder les lignes dans la marge

\newcommand\blfootnote[1]{%
  \begingroup
  \renewcommand\thefootnote{}\footnote{#1}%
  \addtocounter{footnote}{-1}%
  \endgroup
}

\begin{document}

\begin{multicols*}{2}

\setlength{\columnseprule}{0.1pt}
\part{Cohérence temporelle/spatiale}
\section{Temporelle}
Deux ondes de longueurs d'ondes différentes sont incohérentes spatialement : on ajoute les éclairements.
\\
La lumière se propageant sous forme de paquet d'ondes, un train d'onde est émis avec pour caractéristiques
$$\tau \Delta\nu = 1$$
et en posant L sa longueur :
$$L = c \tau$$

Quand on somme deux éclairements issus de deux longeurs d'ondes différentes en utilisant la formule trigo
$$cos(a)+cos(b) = 2cos(\frac{a+b}{2})cos(\frac{a-b}{2})$$
On trouve une formule du type
$$\mathcal{E} = 4\mathcal{E}_0\left[1+cos\left(\frac{2\pi\delta}{\lambda_m}\right)cos\left(\frac{\pi\delta\Delta\lambda}{\lambda_m^2}\right)\right]$$
Le terme cos de gauche est alors interprété comme étant un terme de variation rapide alors que l'enveloppe à gauche possède une variation plus lente : c'est le contraste défini par
$$v = \frac{\mathcal{E}_{max}-\mathcal{E}_{min}}{\mathcal{E}_{max}+\mathcal{E}_{min}}$$


On a souvent besoin d'introduire le sinus cardinal $$sinc(x)=sin(x)/x$$ ayant les propriétés suivantes :\\
$sinc(0) = 1$\\
$sinc(x) = 0 \Leftrightarrow x = m\pi$ avec $m \in \mathbb{Z}^*$ avec des extrémas secondaires placés environ au milieu ie pour $x = (2m + 1)\frac{\pi}{2}$
% \raggedcolumns

\footnotesize{Fiches créés par Alexis Ducarouge et Léo Colisson.}

\end{multicols*}

\end{document}

