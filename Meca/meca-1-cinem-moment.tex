% Compiler avec
% latex fig.tex && convert -alpha Remove -resize 800x600 -depth 8 fig.dvi fig3.png

% \documentclass[convert={density=300,size=1080x800,outext=.png}]{standalone}
\documentclass[9pt]{article}

\usepackage[utf8]{inputenc}   % LaTeX, comprends les accents !
\usepackage[T1]{fontenc}      % Police contenant les caractères français
\usepackage[francais]{babel}  % Placez ici une liste de langues, la
                              % dernière étant la langue principale
\usepackage{lipsum}

\usepackage{multicol}
\usepackage{amsmath}
\usepackage{amsfonts}
\usepackage{amssymb}
\usepackage{calrsfs}
\usepackage[mathscr]{euscript}


\usepackage[paperwidth=600px,
            paperheight=450px,
            top=10px,
            bottom=10px,
            left=10px,
            right=10px]{geometry}

% \usepackage[a4paper]{geometry}% Réduire les marges
\pagestyle{headings}        % Pour mettre des entêtes avec les titres
                              % des sections en haut de page

% \title{Lorem ipsum}           % Les paramètres du titre : titre, auteur, date
% \author{Curabitur \and Elementum}
% \date{}                       % La date n'est pas requise (la date du
                              % jour de compilation est utilisée en son
			      % absence)

% \sloppy                       % Ne pas faire déborder les lignes dans la marge

\begin{document}

\begin{multicols*}{2}
\setlength{\columnseprule}{0.1pt}
\part*{Cinématique}
\section{Définitions}
\subsection{Cinétique}
Résultante cinétique : $\overrightarrow{P}=\sum_im_i\overrightarrow{v_i}$
\\On montre $\overrightarrow{P}=m\overrightarrow{v}_{G/R}$ d'où $\overrightarrow{P^*}=\overrightarrow{0}$.\\

Moment cinétique : $\overrightarrow{L_A}=\sum_i{m_i\overrightarrow{AM}_i\wedge\overrightarrow{v_i}}$\\
Dans $R^*$ le moment cinétique de dépend pas du point considéré, on le note $\overrightarrow{L^*}$.
\subsection{Dynamique}

Résultante dynamique : $\overrightarrow{D}=\sum_im_i\overrightarrow{a_i}=m\overrightarrow{a_{G/R}}=\frac{d\overrightarrow{P}}{dt}$.\\
Moment dynamique : $\overrightarrow{\delta_A}=\sum_i\overrightarrow{AM_i}\wedge\overrightarrow{a_i}$

\subsection{Relation moment cinétique/ moment dynamique}
Cas général $\left(\frac{d\overrightarrow{L_A}}{dt}\right) = \overrightarrow{\delta_A} + \overrightarrow{P}\wedge\overrightarrow{v_{A/R}} = \overrightarrow{\delta_A} + m\overrightarrow{v_{G/R}}\wedge\overrightarrow{v_{A/R}}$.\\
Si A est fixe ou si $A=G$ on a $\frac{d\overrightarrow{L_A}}{dt}=\overrightarrow{\delta_A}$
% \columnbreak
\section{Théorème de Koenig}
\subsection{Cinétique/Dynamique}
\begin{center}
$\boxed{\overrightarrow{L_A}=\overrightarrow{L^*}+\overrightarrow{AG}\wedge\overrightarrow{P}}$
(Avec $\overrightarrow{P}=m\overrightarrow{v_{G/R}}$)
\end{center}
NB : On retrouve la formule de Varignon $\overrightarrow{M_B}=\overrightarrow{M_A}+\overrightarrow{BA} \wedge \overrightarrow{R}$.
\\Et avec la version moment dynamique : $\overrightarrow{\delta_A}=\overrightarrow{\delta^*}+\overrightarrow{AG}\wedge\overrightarrow{D}$

\subsection {Énergie cinétique}
Théorème de Koenig
$$E_c = Ec^* + \frac{1}{2}m\overrightarrow{v_{G/R}}^2$$

\section{Vitesse d'un solide}
Formule de Varignon
$$\boxed{\overrightarrow{v_{A/R}} = \overrightarrow{v_{B/R}} + \overrightarrow{AB}\wedge\overrightarrow{\omega_{S/R}}}$$

Vitesse de glissement : $\overrightarrow{v_g} = \overrightarrow{v_{I_2/R}}-\overrightarrow{v_{I_1/R}}$

Roulement sans glissement : $\overrightarrow{v_{M/R}}=\overrightarrow{\omega_T}\wedge\overrightarrow{IM}$


\section{Solide en rotation}
Moment cinétique : $L_\Delta = \overrightarrow{L_O}\cdot\overrightarrow{u_z} = \omega \iiint_S{HM^2dm} = \omega J_\Delta$
$J_\Delta$ est le moment d'inertie (scalaire).

Energie cinétique :
$$\boxed{E_c = \frac{1}{2}J_\Delta\omega^2 = \frac{1}{2}\overrightarrow{L_O}\cdot\overrightarrow{\omega}}$$

Théorème d'Huygens (HP ?) : Souvent on donne $J_{\Delta_G}$, l'axe $\Delta_G$ passant par G et on souhaite connaitre $J_\Delta$ avec $\Delta$ parallèle à $\Delta_G$. On note $a$ la distance entre ces deux axes, $m$ la masse du solide :
$$J_\Delta = J_{\Delta_G} + ma^2$$
(Preuve en injectant le théorème de Koenig)

\footnotesize{Fiches créés par Alexis Ducarouge et Léo Colisson.}
% \raggedcolumns
\end{multicols*}

\end{document}

