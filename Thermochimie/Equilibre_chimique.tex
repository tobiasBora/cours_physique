% Compiler avec
% latex fig.tex && convert -alpha Remove -resize 800x600 -depth 8 fig.dvi fig3.png

% \documentclass[convert={density=300,size=1080x800,outext=.png}]{standalone}
\documentclass[9pt,twocolumn]{article}

\usepackage[utf8]{inputenc}   % LaTeX, comprends les accents !
\usepackage[T1]{fontenc}      % Police contenant les caractères français
\usepackage[francais]{babel}  % Placez ici une liste de langues, la
                              % dernière étant la langue principale
\usepackage{lipsum}

\usepackage{multicol}
\usepackage{amsmath}
\usepackage{amsfonts}
\usepackage{amssymb}
\usepackage{calrsfs}
\usepackage{mathrsfs}
\usepackage[mathscr]{euscript}


\usepackage[paperwidth=600px,
            paperheight=450px,
            top=10px,
            bottom=10px,
            left=10px,
            right=10px]{geometry}

% \usepackage[a4paper]{geometry}% Réduire les marges
\pagestyle{headings}        % Pour mettre des entêtes avec les titres
                              % des sections en haut de page

% \title{Lorem ipsum}           % Les paramètres du titre : titre, auteur, date
% \author{Curabitur \and Elementum}
% \date{}                       % La date n'est pas requise (la date du
                              % jour de compilation est utilisée en son
			      % absence)

% \sloppy                       % Ne pas faire déborder les lignes dans la marge

\begin{document}

\setlength{\columnseprule}{0.1pt}
\begin{center}
\part*{Équilibre chimique}
\end{center}
\section{Affinité}
On pose l'affinité chimique  : 

$$\>\>\>\boxed{A=-\Delta_rG=-\sum \nu_i \mu_i=-\left(\dfrac{\partial G}{\partial\xi}\right)_{T,P}}$$
\bigbreak

On a alors la relation fondamentale : $\>\boxed{Ad\xi>0}$ ou $\boxed{\Delta_rGd\xi<0}$ qui donne le sens d'évolution de la réaction / l'équilibre.

\bigbreak 
Expression de A : $\boxed{A(T,P,\xi)=A^0(T)-RT ln Q}$

\section{Équilibre chimique}

À l'équilibre, on a $A=0$, on pose alors $K_0=Q_e$ \textbf{la constante d'équilibre}. Avec $Q=\prod_i a_i^{\nu_i}$. On a alors :

$$\boxed{K^0(T) = exp \left( \dfrac{A^0}{RT} \right)= exp \left( \dfrac{-\Delta_rG^0}{RT} \right)}$$

Mais également, la \textbf{loi d'action des masses} :

$$\boxed{K^0(T)=\prod_i (a_i)_{eq}^{\nu_i}}$$

Il est parfois utile de faire des hypothèses simplificatrices (espèce en excès, très peu concentrée) pour effectuer les calculs plus facilement. Il convient de vérifier ces hypothèse en fin de calcul.
\bigbreak

\textbf{Relation isobare de Van't Hoff :}

$$\boxed{\dfrac{d}{dT}(ln K^0)=\dfrac{\Delta_rH^0}{RT^2}}$$

\section{Déplacement d'équilibre}

\subsection{Influence de T}

On a $\dfrac{dA}{dT}=\dfrac{\Delta_rH^0}{T}$.

\smallbreak	
\textbf{Loi de compensation de Van't Hoff :} déplacement de la réaction dans le sens endothermique ($\Delta_rH^0>0$)si T augmente.
\subsection{Influence de P}

On a $dA=-RT\Delta_r\nu_g \dfrac{dP}{P}$.

\smallbreak	
\textbf{Loi de compensation de Le Chatelier :}Toute
augmentation de pression isotherme fait évoluer le système dans le sens d'une diminution de la quantité de gaz.


\subsection{Ajout d'un composant}

On écrit $dA=-RT\dfrac{dQ}{Q}$ avec $\dfrac{dQ}{Q}=\sum \nu_i \dfrac{dn_i}{n_i}$

et alors $n_a\rightarrow n_a+dn_a$ 

mais également : $n_{gaz_{tot}}=\dfrac{n_a}{x_a}$ ainsi que : $dn_{gaz_{tot}}=dn_a$

Puis il reste à déterminer le signe de $dA$ en fonction de $x_a$.
\vfill
\footnotesize{Fiches créés par Alexis Ducarouge et Léo Colisson.}


\end{document}

