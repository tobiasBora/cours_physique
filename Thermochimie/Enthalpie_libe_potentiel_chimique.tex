% Compiler avec
% latex fig.tex && convert -alpha Remove -resize 800x600 -depth 8 fig.dvi fig3.png

% \documentclass[convert={density=300,size=1080x800,outext=.png}]{standalone}
\documentclass[9pt,twocolumn]{article}

\usepackage[utf8]{inputenc}   % LaTeX, comprends les accents !
\usepackage[T1]{fontenc}      % Police contenant les caractères français
\usepackage[francais]{babel}  % Placez ici une liste de langues, la
                              % dernière étant la langue principale
\usepackage{lipsum}

\usepackage{multicol}
\usepackage{amsmath}
\usepackage{amsfonts}
\usepackage{amssymb}
\usepackage{calrsfs}
\usepackage{mathrsfs}
\usepackage[mathscr]{euscript}


\usepackage[paperwidth=600px,
            paperheight=450px,
            top=10px,
            bottom=10px,
            left=10px,
            right=10px]{geometry}

% \usepackage[a4paper]{geometry}% Réduire les marges
\pagestyle{headings}        % Pour mettre des entêtes avec les titres
                              % des sections en haut de page

% \title{Lorem ipsum}           % Les paramètres du titre : titre, auteur, date
% \author{Curabitur \and Elementum}
% \date{}                       % La date n'est pas requise (la date du
                              % jour de compilation est utilisée en son
			      % absence)

% \sloppy                       % Ne pas faire déborder les lignes dans la marge

\begin{document}

\setlength{\columnseprule}{0.1pt}


\part*{Enthalpie libre}
\section{Définition}
\begin{center}
$\boxed{G=U+PV-TS=H-TS}$
\end{center}

On a $\Delta G\leq W_{autre}$, l'égalité étant atteinte dans le cas réversible.
\medbreak
Forme différentielle : $\boxed{dG=-SdT+VdP}$
\medbreak
Donc : $S=-\left(\frac{\partial G}{\partial T}\right) _P$ et $V=-\left(\frac{\partial G}{\partial P}\right) _T$


\section{Relation de Gibbs-Helmholtz}
\begin{center}

$\displaystyle{\dfrac{\partial}{\partial T}\left( \dfrac{G}{T} \right) _P=\dfrac{-H}{T^2}}$
\end{center}


\section{Enthalpie libre d'un gaz parfait}
$\boxed{G(P,T)=G^0(T)+nRT ln\frac{P}{P^0}}$

\section{Enthalpie libre d'une phase condensée}
$\boxed{G(P,T)=G^0(T)+(P-P^0)V^0}$

\section{Énergie libre}

\begin{center}
$F=U-TS$
\end{center}
On a : $\Delta F\leq W$, l'égalité étant atteinte dans le cas réversible.
Forme différentielle : $dF=-SdT-PDV$



\part*{Potentiel Chimique}
\section{Définition}

$\boxed{\mu_i=\left(\dfrac{\partial G}{\partial n_i}\right)_{T,P,n_{j\neq i}}}$ et ainsi : $dG=-SdT+VdP+\sum\limits_{\substack{i=0}}^{n}  \mu_i dn_i$

Relation dite d'Euler : $\boxed{G=\sum\limits_{\substack{i=0}}^{n}  \mu_i n_i}\>\>$ d'où $s_i=-\left(\frac{\partial \mu_i}{\partial T}\right) _{P,n_j}$ 
\bigbreak
Forme différentielle : $\boxed{d\mu=-s_mdT+v_mdP}$ pour un corps pur.
\bigbreak

On a $\Delta G\leq0$ donc la réaction a lieu \textbf{dans le sens du plus petit potentiel chimique.} Équilibre ssi $\mu_l=\mu_v$.

\section{Potentiel chimique pour un gaz parfait}

$\boxed{\mu(P,T)=\mu^0(T)+RT ln\frac{P}{P^0}}$

\section{Mélange de gaz parfaits}

$\boxed{\mu_i(P_i,T)=\mu_i^0(T)+RT ln\frac{P_i}{P^0}}\>\>$ avec $P_i=x_iP=\dfrac{n_i}{n_{tot}}P$


\section{Mélange de phase condensée}

$\boxed{\mu_i(P,T)=\mu^0(T)+v_m^0(P-P^0)+RT ln x_i}$ 

avec ici $x_i=a_i$ qui vaut 1 si : corps pur et une seule phase.
\bigbreak

Cas d'un soluté (mélange idéal) : $\boxed{\mu_i(P,T)=\mu^0(T)+RT ln \dfrac{c_i}{c^0}}$ 

% \raggedcolumns
\vfill
\footnotesize{Fiches créés par Alexis Ducarouge et Léo Colisson.}


\end{document}

