% Compiler avec
% latex fig.tex && convert -alpha Remove -resize 800x600 -depth 8 fig.dvi fig3.png

% \documentclass[convert={density=300,size=1080x800,outext=.png}]{standalone}
\documentclass[9pt,twocolumn]{article}

\usepackage[utf8]{inputenc}   % LaTeX, comprends les accents !
\usepackage[T1]{fontenc}      % Police contenant les caractères français
\usepackage[francais]{babel}  % Placez ici une liste de langues, la
                              % dernière étant la langue principale
\usepackage{lipsum}

\usepackage{multicol}
\usepackage{amsmath}
\usepackage{amsfonts}
\usepackage{amssymb}
\usepackage{calrsfs}
\usepackage{mathrsfs}
\usepackage[mathscr]{euscript}


\usepackage[paperwidth=600px,
            paperheight=450px,
            top=10px,
            bottom=10px,
            left=10px,
            right=10px]{geometry}

% \usepackage[a4paper]{geometry}% Réduire les marges
\pagestyle{headings}        % Pour mettre des entêtes avec les titres
                              % des sections en haut de page

% \title{Lorem ipsum}           % Les paramètres du titre : titre, auteur, date
% \author{Curabitur \and Elementum}
% \date{}                       % La date n'est pas requise (la date du
                              % jour de compilation est utilisée en son
			      % absence)

% \sloppy                       % Ne pas faire déborder les lignes dans la marge

\begin{document}

\setlength{\columnseprule}{0.1pt}
\begin{center}
\part*{Thermodynamique}
\end{center}
\section{Bases}
Loi des gaz parfaits : $\boxed{PV=nRT}$  $\>\>\>$ avec $n=\dfrac{N}{N_A}$ et $k_B=\dfrac{R}{N_A}$

Loi de Van der Waals : $(P+\dfrac{n^2a}{v^2})(V-nb)=nRT$ 

\medbreak
Coefficient de dilatation isobare : 
$\alpha=\frac{1}{V} \left( \frac{\partial V}{\partial T} \right) _P$

Coefficient de compressibilité isotherme : 
$\chi _T=\frac{-1}{V} \left( \frac{\partial V}{\partial P} \right) _T$

\section{Premier principe}

Forme différentielle : $\boxed{dE=\delta W+\delta Q}$ 
\smallbreak
avec $dE=dU+dE_p+dE_c$
\medbreak
Loi de Joule : $dU=C_v dT$ avec $ C_v=\dfrac{nR}{\gamma-1}$


\bigbreak

Forme système ouvert : $\boxed{dH=\delta Q} \>\>\>\>\>$ avec $H=U+PV$
\smallbreak
Loi de Joule : $dH=C_p dT$ avec $ C_p=\dfrac{\gamma nR}{\gamma-1}$ 

Relations utiles : $\gamma =\frac{C_p}{C_v}$ mais aussi $C_p-C_v=nR$


\section{Second principe}

Forme différentielle : $\boxed{dS=\delta S_e+\delta S_c}$ 
\medbreak
Entropie échangée : $S_e=\frac{Q}{T}$

Entropie créée : $S_c\geq0$ (égalité si réversibilité)

\section{Loi de Laplace}

Système adiabatique + réversible = isentropique.

$\>\>\>\>\>\>\>\>\>\>\>\>PV^{\gamma}=cst  \>\>\>\>\>\>\> TV^{\gamma-1}=cst   \>\>\>\>\>\>\> T^{\gamma}P^{1-\gamma}=cst$


\section{Identité thermodynamique}

Forme différentielle : $\boxed{dH=TdS+PdV}$
\bigbreak
D'où (à l'image de la loi de Laplace) :

$\Delta S=C_v ln \dfrac{PV^{\gamma}}{P_0 V_0^{\gamma}}=C_v ln \dfrac{TV^{\gamma-1}}{T_0 V_0^{\gamma-1}}=C_v ln \dfrac{T^{\gamma}P^{1-\gamma}}{T_0^{\gamma} !P_0^{1-\gamma}}$ 

% \raggedcolumns

\vfill
\footnotesize{Fiches créés par Alexis Ducarouge et Léo Colisson.}


\end{document}

