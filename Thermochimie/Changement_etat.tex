% Compiler avec
% latex fig.tex && convert -alpha Remove -resize 800x600 -depth 8 fig.dvi fig3.png

% \documentclass[convert={density=300,size=1080x800,outext=.png}]{standalone}
\documentclass[9pt,twocolumn]{article}

\usepackage[utf8]{inputenc}   % LaTeX, comprends les accents !
\usepackage[T1]{fontenc}      % Police contenant les caractères français
\usepackage[francais]{babel}  % Placez ici une liste de langues, la
                              % dernière étant la langue principale
\usepackage{lipsum}

\usepackage{multicol}
\usepackage{amsmath}
\usepackage{amsfonts}
\usepackage{amssymb}
\usepackage{calrsfs}
\usepackage{mathrsfs}
\usepackage[mathscr]{euscript}


\usepackage[paperwidth=600px,
            paperheight=450px,
            top=10px,
            bottom=10px,
            left=10px,
            right=10px]{geometry}

% \usepackage[a4paper]{geometry}% Réduire les marges
\pagestyle{headings}        % Pour mettre des entêtes avec les titres
                              % des sections en haut de page

% \title{Lorem ipsum}           % Les paramètres du titre : titre, auteur, date
% \author{Curabitur \and Elementum}
% \date{}                       % La date n'est pas requise (la date du
                              % jour de compilation est utilisée en son
			      % absence)

% \sloppy                       % Ne pas faire déborder les lignes dans la marge

\begin{document}

\setlength{\columnseprule}{0.1pt}
\begin{center}
\part*{Changement d'état}
\end{center}
\section{Variance}

\textbf{Variance :} nombre de paramètres intensifs qu'il faut fixer pour déterminer l'état du système.
$$v=C+2-\phi-r-q$$

 
 - $C$ : nombre de constituants
 
 - $\phi$ : nombre de phases distinctes (corps miscibles : une pour les gaz, une pour les liquides ; corps non miscibles : chacun une)
 
 - $r$ : nombre d'équilibres chimiques indépendants
 
 - $q$ : nombre d'autres relations (stœchiométrie par exemple)
 
 \section{Diagramme / Formule de Clapeyron}
 
 C'est un diagramme $P=f(V)$, composé des courbes de séparation de phase (\textbf{ébullition}, \textbf{rosée}) et des isothermes d'Andrews.
 \bigbreak
\textbf{ Enthalpie massique de changement d'état :} variation d'enthalpie lors du changement d'état de la phase la plus ordonnée vers la moins ordonnée (1 kg, T et P fixé).
$$ \boxed{L_{1\rightarrow 2}=T\dfrac{dP_{eq}}{dT}(v_2-v_1)}$$ 
 
 \section{Fonctions d'état}

En raisonnant sur le diagramme (P,V) : \textbf{ne pas la courbe de rosée : capacité de la vapeur saturante inconnue}.
\medbreak
H : $ \>\>\> H_M = H_0 + m x_v L_v(T_M) + m c_l (T_M - T_0)$
 \medbreak
U : $ \>\>\> U_M = U_0 + m x_v L_v(T_M)  - m P (v_v-v_l) x_v+ m c_l (T_M - T_0)$
\smallbreak
 S : $ \>\>\> S_M = S_0 + m x_v \dfrac{L_v(T_M)}{T_M} + m c_l ln \left(\dfrac{T_M}{T_0}\right)$

\vfill
\footnotesize{Fiches créés par Alexis Ducarouge et Léo Colisson.}


\end{document}

