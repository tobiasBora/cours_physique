% Compiler avec
% latex fig.tex && convert -alpha Remove -resize 800x600 -depth 8 fig.dvi fig3.png

% \documentclass[convert={density=300,size=1080x800,outext=.png}]{standalone}
\documentclass[9pt,twocolumn]{article}

\usepackage[utf8]{inputenc}   % LaTeX, comprends les accents !
\usepackage[T1]{fontenc}      % Police contenant les caractères français
\usepackage[francais]{babel}  % Placez ici une liste de langues, la
                              % dernière étant la langue principale
\usepackage{lipsum}

\usepackage{multicol}
\usepackage{amsmath}
\usepackage{amsfonts}
\usepackage{amssymb}
\usepackage{calrsfs}
\usepackage{mathrsfs}
\usepackage[mathscr]{euscript}


\usepackage[paperwidth=600px,
            paperheight=450px,
            top=10px,
            bottom=10px,
            left=10px,
            right=10px]{geometry}

% \usepackage[a4paper]{geometry}% Réduire les marges
\pagestyle{headings}        % Pour mettre des entêtes avec les titres
                              % des sections en haut de page

% \title{Lorem ipsum}           % Les paramètres du titre : titre, auteur, date
% \author{Curabitur \and Elementum}
% \date{}                       % La date n'est pas requise (la date du
                              % jour de compilation est utilisée en son
			      % absence)

% \sloppy                       % Ne pas faire déborder les lignes dans la marge

\begin{document}

\setlength{\columnseprule}{0.1pt}
\begin{center}
\part*{Grandeurs standard}
\end{center}
\section{Grandeur standard de réaction}

$$\boxed{ \Delta _r X^0(T,\xi) = \sum\limits_i \nu_i X^0_{m,i}(T,\xi) = \left(\dfrac{\partial X^0}{\partial \xi} \right) }$$

Attention ne pas confondre : 

$\Delta _r X^0(T,\xi) =\left(\dfrac{\partial X^0}{\partial \xi} \right) \>\>\>$ et $\>\>\> \Delta X = X(\xi_2)-X(\xi_1) = \displaystyle{\int_{\xi_1}^{\xi_2} \Delta_r X d\xi}$
\medbreak
Ainsi : $\boxed{\Delta H_{12}=(\xi_2-\xi_1) \Delta_r H^0}$

\section{Lois de Kirchhoff}
Relations à intégrer entre 2 températures :
\smallbreak
Pour H : $\>\>\>\boxed{\dfrac{d(\Delta_r H^0)}{dT}=\sum\limits_i\nu_i C_{P,m,i}^0(T) = \Delta_rC_P^0(T)}$ 
\smallbreak
Pour S : $\>\>\>\>\boxed{\dfrac{d(\Delta_r S^0)}{dT}=\dfrac{1}{T}\sum\limits_i\nu_i C_{P,m,i}^0(T) = \dfrac{1}{T}\Delta_rC_P^0(T)}$

\smallbreak

\textbf{Approximation d'Ellingham :} tant qu'il n'y a pas de changement d'état, on suppose $\Delta_rH^0$ et $\Delta_rS^0$ \textbf{indépendants de la température} (mais pas $\Delta_rG^0$ !).

\section{Grandeurs standard de formation}
Relation valable pour \textbf{une mole} de corps à partir des \textbf{corps simples} dans leur \textbf{état standard de référence}.
\medbreak
\textbf{Loi de Hess :} $\>\>\>\boxed{\Delta_rX^0(T)=\sum\limits_i\nu_i\Delta_fX^0(T)}$

\section{Relations entre grandeurs standard}

Partant de : $\>\boxed{G=-TS+H}\>$ on obtient :

$$\boxed{\Delta_rG^0(T)=\Delta_rH^0(T)-T\Delta_rS^0(T)}$$

$$\boxed{\dfrac{d\Delta_rG^0}{dT}=-\Delta_rS^0(T)}$$

Relation de Gibbs-Hedlmholtz :
$$\boxed{\dfrac{d}{dT}\left(\dfrac{\Delta_rG^0(T)}{T}\right)=-\dfrac{\Delta_rH^0(T)}{T^2}}$$

On a également le lien avec les grandeurs de changement d'état : 

$\>\>\>\>\>\>\>\>\>\>\>\>\>\>\>\>\>\>\>\>\>\>\>\>\>\Delta_rH^0=L_{f,m}\>\>$ ou $\>\>\Delta_rS^0=\frac{L_{f,m}}{T_f}$


\bigbreak

Pour trouver l'énergie $U$ :
\smallbreak
$\Delta_rU^0=\Delta_rH^0 - RT \sum \nu_{i,gaz}=\Delta_rH^0 - RT \Delta_r\nu_{gaz}$

\vfill
\footnotesize{Fiches créés par Alexis Ducarouge et Léo Colisson.}

\end{document}

