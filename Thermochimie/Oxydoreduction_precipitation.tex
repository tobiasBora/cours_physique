% Compiler avec
% latex fig.tex && convert -alpha Remove -resize 800x600 -depth 8 fig.dvi fig3.png

% \documentclass[convert={density=300,size=1080x800,outext=.png}]{standalone}
\documentclass[9pt,twocolumn]{article}

\usepackage[utf8]{inputenc}   % LaTeX, comprends les accents !
\usepackage[T1]{fontenc}      % Police contenant les caractères français
\usepackage[francais]{babel}  % Placez ici une liste de langues, la
                              % dernière étant la langue principale
\usepackage{lipsum}

\usepackage{multicol}
\usepackage{amsmath}
\usepackage{amsfonts}
\usepackage{amssymb}
\usepackage{calrsfs}
\usepackage{mathrsfs}
\usepackage[mathscr]{euscript}


\usepackage[paperwidth=600px,
            paperheight=450px,
            top=10px,
            bottom=10px,
            left=10px,
            right=10px]{geometry}

% \usepackage[a4paper]{geometry}% Réduire les marges
\pagestyle{headings}        % Pour mettre des entêtes avec les titres
                              % des sections en haut de page

% \title{Lorem ipsum}           % Les paramètres du titre : titre, auteur, date
% \author{Curabitur \and Elementum}
% \date{}                       % La date n'est pas requise (la date du
                              % jour de compilation est utilisée en son
			      % absence)

% \sloppy                       % Ne pas faire déborder les lignes dans la marge

\begin{document}

\setlength{\columnseprule}{0.1pt}
\part*{Réactions d'oxydo-réduction}
\section{Demi-équations rédox}
Partant du couple : Ox/Red (capte/cède), pour équilibrer :

 - équilibrer l'élément principal

 - équilibrer $O$

 - équilibrer $H$ (avec $H^+$ ou $OH^-$ selon la solution)

 - équilibrer les charge avec des électrons
 
 \bigbreak
 On peut aussi compter le \textbf{nombre d'oxydation} : charge que doit avoir chaque élément principal pour ramener la particule à la neutralité électrique, sachant que : 

 - il faut prendre en compte la charge moléculaire

 - $O$ déséquilibre de 2- et $H$ de 1+
 \bigbreak
 Pour l'\textbf{équation complète} : utiliser le ppcm du nombre d'électron des demi-équations.

\section{Constante de réaction}

$$\boxed{K=\dfrac{\prod [Produits]^{\nu_i}}{\prod [Réactifs]^{\mu_i}}}$$
\section{Formule de Nersnt}
$$\boxed{E=E^0+\dfrac{0.006}{n_{e^-\>echanges}}log\>\dfrac{[Ox]^{\nu_i}}{[Red]^{\mu_i}}}$$ 
À l'équilibre $E_{Ox}=E_{Red}$, d'où: $K=10^{\frac{ppcm(n_{Ox},n_{Red})}{0.06}(E^0_{Ox}-E^0_{Red})}$

\section{Autre approche des potentiels}

Pour déterminer un potentiel par combinaison linéaires d'autres potentiel connus :

$\>\>\>\>\>\>\>\>\>\>\>\>\>\>\>\>\>\>\>\>\>\>\>\>\>\>\>\>\>\>\boxed{E^0=-\dfrac{\Delta_rG^0}{n\mathcal{F}} = \dfrac{A^0}{n\mathcal{F}}}$ avec $\mathcal{F}=96500C$
\medbreak
Ou une constante d'équilibre par :
$\>\>\>\>\boxed{\Delta_rG^0=-RT\>lnK^0}$


\section{Diagramme E-pH de l'eau}

$H_2O \>$(ici $H^+$) oxydant : $\>\>\> H^+ + e^-= \frac{1}{2}H_2$ avec $E^0=0V$
\smallbreak
$H_2O$ réducteur : $\>\>\> O_2+4e^-+4H^+=2H_2O$ avec $E^0=1,23V$
\medbreak
Autres diagramme pH :

 - droite verticale : écrire $K=fct([H_3O^+]) \>\Rightarrow\>pH=cst$
 
 - droite oblique : écrire la loi de Nernst (pente fonction du $pH$).
 
 
\part*{Réactions de précipitation}
\section{Produit de solubilité}

Réaction : $ AB = A^+ + B^-$. On définit le produit de solubilité :

$$\boxed{K_s=[A^+][B^-]}$$

\medbreak
Apparition du solide ssi $[A^+][B^-]>K_s$

\section{Hypothèses simplificatrices (à vérifier)}
- $K < 10^{-3}$ : réaction peu avancée : $\xi\ll C_0 v_0$

- $K > 10^{3}$ : réaction totale : $\xi\simeq C_{0_{limitant}} v_0$

- volume $v\gg v_0$

\vfill
\footnotesize{Fiches créés par Alexis Ducarouge et Léo Colisson.}

\end{document}

