% Compiler avec
% latex fig.tex && convert -alpha Remove -resize 800x600 -depth 8 fig.dvi fig3.png

% \documentclass[convert={density=300,size=1080x800,outext=.png}]{standalone}
\documentclass[9pt,twocolumn]{article}

\usepackage[utf8]{inputenc}   % LaTeX, comprends les accents !
\usepackage[T1]{fontenc}      % Police contenant les caractères français
\usepackage[francais]{babel}  % Placez ici une liste de langues, la
                              % dernière étant la langue principale
\usepackage{lipsum}

\usepackage{multicol}
\usepackage{amsmath}
\usepackage{amsfonts}
\usepackage{amssymb}
\usepackage{calrsfs}
\usepackage{mathrsfs}
\usepackage[mathscr]{euscript}


\usepackage[paperwidth=600px,
            paperheight=450px,
            top=10px,
            bottom=10px,
            left=10px,
            right=10px]{geometry}

% \usepackage[a4paper]{geometry}% Réduire les marges
\pagestyle{headings}        % Pour mettre des entêtes avec les titres
                              % des sections en haut de page

% \title{Lorem ipsum}           % Les paramètres du titre : titre, auteur, date
% \author{Curabitur \and Elementum}
% \date{}                       % La date n'est pas requise (la date du
                              % jour de compilation est utilisée en son
			      % absence)

% \sloppy                       % Ne pas faire déborder les lignes dans la marge

\begin{document}

\setlength{\columnseprule}{0.1pt}
\part*{Réactions acido-basiques}
\section{pH/pKe}
$$\boxed{pH=-log\>[H_3O^+]}$$
$$K_e=[H_3O^+][HO^-]$$
$$pK_e=14=-logK_e=pH+pOH$$


\section{Ka/pKa}

Couple : $HA/A^-$   donne la réaction : $HA+H_2O = H_3O^+ + A^-$

 on a alors : $K_a=\dfrac{[A-][H_3O^+]}{[HA]}$
 \medbreak 
 On définit : $pK_a=-log\>K_a$ d'où $K_a=10^{-pK_a}$
 
 \medbreak
 On peut alors en déduire : $\boxed{pH=pK_a+log\>\dfrac{[A^-]}{[HA]}}$
 
 \medbreak
 
 Pour tracer un axe $pK_a$ : axe avec les $pK_a$ et leur couple associé (acides en haut, bases en bas).
 
 "Gamma droit" : $K_a=\dfrac{K_{a_{gauche}}}{K_{a_{droite}}}=10^{pK_{a_{droite}}-pK_{a_{gauche}}}$
\section{Méthode de la réaction prépondérante}

- S'occuper des réactions quantitatives ($K_a>10^3$).

$\>\>\>\>\Rightarrow$ Tracer un $pK_a$ puis méthode du "gamma droit".

- Réaction quantitative : produits complètement consommés.

$\>\>\>\>\Rightarrow$ Nouveau bilan des espèces en présence.

- À la première réaction prépondérante (non totale), on calcul l'état final grâce au $K$.



\part*{Réactions de complexation}
\section{Réaction}
$M_{etal}+L_{igant} = ML_{n_{complexe}}$  n appelé indice de coordinence.
\medbreak
On définit : $Kf_n=\dfrac{[ML_n]}{[ML_{n-1}][L]}$ et $Kd_n=\dfrac{[ML_{n-1}][L]}{ML_n}$


\section{pL}
On définit également $pL=-log\>[L]$ et $\beta_n=\dfrac{[ML_n]}{[M][L]^n}$

d'où : $\boxed{p\beta_n=log\>\dfrac{[ML_n]}{[M]}+n\>pL}$
\bigbreak
On peut alors, de même que pour les acido-basiques, définir des droites en $pL$ ou $pKd$.

\vfill
\footnotesize{Fiches créés par Alexis Ducarouge et Léo Colisson.}

\end{document}

